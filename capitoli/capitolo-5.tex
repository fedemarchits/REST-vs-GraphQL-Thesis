% !TEX encoding = UTF-8
% !TEX TS-program = pdflatex
% !TEX root = ../tesi.tex

%**************************************************************
\chapter{Analisi comparativa dei protocolli REST e GraphQL}
\label{cap:analisi-comparativa}
%**************************************************************
\intro{In questo capitolo verrà illustrata l'analisi comparativa tra i due protocolli REST e GraphQL, sia dal punto di vista teorico che da quello pratico .}\\
%**************************************************************
%\section{Analisi comparativa teorica}
%\label{sec:analisi-comparativa-teorica}
%Per analisi comparativa teoria s'intende tutti quegli aspetti che differenziano i due protocolli GraphQL e REST
\section{Introduzione}
REST è stato ed è tutt'oggi lo standard più seguito per la realizzazione delle Web Api, tuttavia dopo l'uscita di GraphQL il suo monopolio è stato messo in crisi. Infatti GraphQL ha portato con se delle interessanti soluzioni per molti dei problemi e dei vincoli di REST.\\
Nel seguente capitolo verranno analizzati nel dettaglio e paragonati i due protocolli, sotto tutti i punti di vista, mettendo in risalto vantaggi e svantaggi di ciascuno.






% Spiegazione di cosa si intende per analisi comparativa teorica: analisi % comparativa che va a comparare gli aspetti prettamente teorici, come ad esempio:
% \begin{itemize}
%  \item overfetching e underfetching;
%  \item un protocollo è una sorta di standard/stile architetturale (restfull api)     mentre l'altro un linguaggio di query fortemente tipizzato (REST invece non è safe dal punto di vista dei tipi);
%  \item come sfruttano il protocollo http (stati risposte http, endpoint multipli vs singolo, ecc...);
%  \item manuntenibilità nel tempo;
%  \item documentazione (GraphQL si autodocumenta, REST no);
%  \item meccanismo di caching integrato (mancante in GraphQL);
%  \item formati output di risposta (GraphQL --> JSON, REST --> JSON, XML, YAML);
% \end{itemize}
% \section{Analisi comparativa sui casi d'uso}
% Analisi comparativa basata su:
% \begin{itemize}
%  \item differenze nell'analisi e progettazione iniziale delle API;
%  \item differenze durante lo sviluppo delle API dal punto di vista di BE e FE(anche legate agli strumenti utilizzati ad es. Spring Data REST vs Spring GraphQL);
%  \item differenze prestazionali (utilizzato tool K6 per load test);
% \end{itemize}2
