% !TEX encoding = UTF-8
% !TEX TS-program = pdflatex
% !TEX root = ../tesi.tex
%**************************************************************
\chapter{Conclusioni}
\label{conclusioni}
\intro{Nel seguente capitolo vengono tratte le conclusioni sull'esperienza di stage e sul lavoro svolto}\\
%**************************************************************
\section{Consuntivo finale}
La pianificazione delle ore realizzata prima di svolgere lo stage, riportata al punto \ref{pianificazione}, è stata rispettata parzialmente. I principali errori durante la pianificazione sono stati fatti nei seguenti casi:
\begin{itemize}
  \item \textbf{Studio delle tecnologie}: lo studio delle tecnologie è stato inizialmente pianificato su metà del tempo a disposizione per lo stage, ovvero le prime quattro settimane; sono state sufficienti tre settimane per apprendere le tecnologie da utilizzare durante lo sviluppo e la migrazione del prototipo e dell'applicativo aziendale;
  \item \textbf{Sviluppo e migrazione del prototipo e SushiLab}: anche per lo sviluppo e la migrazione del prototipo e di SushiLab erano state previste quattro settimane, tuttavia il tempo previsto non è stato sufficiente poiché il lavoro è stato svolto in cinque settimane;
\end{itemize}
Fortunatamente i due punti si sono bilanciati, dunque è stato possibile sfruttare la settimana in più prevista per lo studio delle tecnologie per completare la migrazione dell'applicativo SushiLab.
%**************************************************************
\section{Raggiungimento degli obiettivi}
Gli obiettivi previsti e riportati al punto \ref{obiettivi-previsti} sono stati pienamente raggiunti. Il prototipo è stato realizzato in maniera funzionante sia nella versione REST che in quella GraphQL; anche l'applicativo aziendale SushiLab è stato migrato con successo fornendo all'azienda l'applicativo nella sua nuova versione GraphQL. Il documento sull'analisi di comparazione tra i due protocolli è stato concluso. L'obiettivo facoltativo riguardante la migrazione di un ulteriore applicativo aziendale oltre a SushiLab non è stato realizzato, al suo posto si è preferito eseguire un' analisi prestazionale dei due protocolli.
Di seguito gli obbiettivi soddisfatti:
\begin{itemize}
  \item \underline{O01}: è stato realizzato lo studio con relativa relazione sulle tecnologie REST e GraphQL;
  \item \underline{O02}: sono stati appresi i framework Spring e Angular;
  \item \underline{O03}: è stato sviluppato e migrato il prototipo;
  \item \underline{O04}: è stata effettuata la migrazione da REST a GraphQL dell'applicativo aziendale SushiLab;
  \item \underline{D01}: è stata realizzata la relazione riguardante l'analisi comparativa tra REST e GraphQL;
  \item \underline{F01}: non è stata realizzata la migrazione di un ulteriore applicativo aziendale; al posto di questo obiettivo è stato realizzato l'obiettivo inizialmente non previsto \underline{F02};
  \item \underline{F02}: è stata realizzata un'analisi prestazionale;
\end{itemize}
%**************************************************************
\section{Conoscenze acquisite}
Durante lo stage sono state acquisite o rassodate le seguenti conoscenze:
\begin{itemize}
  \item \textbf{Sviluppo Backend}: è stato compreso come sviluppare un server di backend con le seguenti tecnologie apprese:
  \begin{itemize}
    \item Spring Boot e i vari altri moduli utilizzati ad esempio: Spring Data JPA, Spring Data REST, Spring GraphQL;
    \item JUnit5 per eseguire test di unità sui metodi del server;
    \item REST e GraphQL per la realizzazione di API;
  \end{itemize}
  \item \textbf{Sviluppo Frontend}: è stato compreso come sviluppare un client di frontend attraverso l'utilizzo delle seguenti tecnologie apprese:
  \begin{itemize}
    \item Angular per la realizzazione di una Single Web Application;
    \item Jasmine per eseguire i test di unità sui servizi di Angular;
  \end{itemize}
  \item \textbf{REST e GraphQL}: sono stati approfonditi e compresi a fondo i due protocolli di data fetching al fine di realizzare un'analisi comparativa approfondita;
  \item \textbf{Organizzazione del lavoro}: è stato appreso come effettuare una buona distribuzione del lavoro da svolgere.
\end{itemize}
%**************************************************************
\section{Valutazione personale}
L'esperienza di stage realizzata nel periodo che va dal 05/09/2022 al 28/10/2022 è stata molto istruttiva per diversi motivi. Innanzitutto è stato possibile affrontare tecnologie sconosciute sotto la supervisione di personale esperto nel settore, poi è stato possibile vivere un'esperienza lavorativa all'interno di un azienda ben organizzata e dunque comprenderne le dinamiche interne. É stata istruttiva anche perché mi ha permesso di affrontare e risolvere nuove problematiche, di organizzarmi il lavoro in modo da riuscire a svolgerlo con successo e infine di assaporare le soddisfazioni che si provano a completare con successo il risultato di due mesi di lavoro.\\ \\
Valuto positivamente la mia esperienza poiché ho saputo adeguarmi all'ambiente lavorativo rispettandone i ritmi e le richieste. Sono soddisfatto del lavoro svolto e della opportunità che è stata concessa dall'Università di Padova.
