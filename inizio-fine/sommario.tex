% !TEX encoding = UTF-8
% !TEX TS-program = pdflatex
% !TEX root = ../tesi.tex

%**************************************************************
% Sommario
%**************************************************************
\cleardoublepage
\phantomsection
\pdfbookmark{Sommario}{Sommario}
\begingroup
\let\clearpage\relax
\let\cleardoublepage\relax
\let\cleardoublepage\relax

\chapter*{Sommario}
Le moderne Web Application adottano un disaccoppiamento stretto tra client e backend.\\\\
Soluzioni architetturali come REST garantiscono di poter realizzare API di utilizzo generale fruibili da svariati client, dai browser sino alle più moderne applicazioni mobili.\\\\
Pur essendo REST lo standard de-facto per la scrittura di Web API, esso presenta alcune debolezze che nuovi strumenti sorti negli utlimi anni cercano di superare.\\\\
GraphQL è certamente una delle più recenti e popolari tecnologie che il mercato dell'Information Technology ci mette a disposizione per la realizzazione di Web API.\\\\
Obbiettivo di questa tesi è quello di realizzare una comparazione tra le tecnologie di data fetching REST e GraphQL individuando i vantaggi nell'adozione di una tecnologia rispetto all'altra; le caratteristiche dei due approcci emergono chiaramente durante lo sviluppo e migrazione di una applicazione da un approccio all'altro.
\endgroup

\vfill
